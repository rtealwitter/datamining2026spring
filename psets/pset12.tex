\documentclass{article}
\usepackage[utf8]{inputenc}
\usepackage[a4paper, total={6in, 8in}]{geometry}
\usepackage{amsmath, amsfonts}
\usepackage{hyperref, graphicx}
\usepackage{tikz, bbm, mathtools}

\DeclareMathOperator*{\argmin}{arg\,min}
\DeclareMathOperator*{\argmax}{arg\,max}

\title{CSCI 145 Problem Set 12}
\author{} % TODO: Put your name here
\date{\today}

\begin{document}

\maketitle

\subsection*{Submission Instructions}

Please upload \textit{your} work by
\textbf{11:59pm Sunday November 23, 2025.}
\begin{itemize}
\item You are encouraged to discuss ideas
and work with your classmates. However, you
\textbf{must acknowledge} your collaborators
at the top of each solution on which
you collaborated with others 
and you \textbf{must write} your solutions
independently.
\item Your solutions to theory questions must
be written legibly, or typeset in LaTeX or markdown.
If you would like to use LaTeX, you can import the source of this document (available from the course webpage) to Overleaf.
\item I recommend that you write your solutions to coding questions in a Jupyter notebook using Google Colab.
\item You should submit your solutions as a \textbf{single PDF} via the assignment on Gradescope.
\end{itemize}

\noindent
\textbf{Grading:} The point of the problem set is for \textit{you} to learn. To this end, I hope to disincentivize the use of LLMs by \textbf{not} grading your work for correctness. Instead, you will grade your own work by comparing it to my solutions. This self-grade is due the \textbf{Monday} \textit{after} the problem set is due, also on Gradescope.

\newpage
\section*{Problem 1: Mean Estimation}

Consider $n$ real numbers $y_1, \ldots, y_n$.
The mean is defined as
\begin{align}
    \bar{y} = \frac1n \sum_{i=1}^n y_i.
\end{align}
In this problem, we will estimate the mean with a subset of $m$ samples $S \subseteq \{1, \ldots, n\}$.

Suppose we have covariates $\mathbf{x}_i \in \mathbb{R}^d$ for each point $i$.
It is expensive to compute $y_i$, but we know how to learn a function $f: \mathbb{R}^d \to \mathbb{R}$ so that $f(\mathbf{x}_i) \approx y_i$, from cheaper training data.
Define the mean of the function so that
\begin{align}
    \bar{f} = \frac1n \sum_{i=1}^nf(\mathbf{x}_i).
\end{align}
Our estimate for the mean will be
\begin{align}
    \hat{y} = \bar{f} + \frac1{m} \sum_{i \in S} (y_i - f(\mathbf{x}_i)).
\end{align}

\noindent \textbf{Note:} For parts A and B, assume your samples are drawn \textit{without} replacement.

\subsection*{Part A: Unbiased}

Show that $\mathbb{E}[\hat{y}] = \bar{y}.$

\subsection*{Part B: Variance}

Carefully derive an upper bound on $\text{Var}(\hat{y})$.

\subsection*{Part C: Empirical Estimation}

Load a regression dataset of your choice, and train a function $f$ on the training set.
Now, we'll compare several estimators for the mean:
\begin{itemize}
    \item Monte Carlo with replacement: $ \frac1m \sum_{i \in S} y_i$ where samples in $S$ are drawn \textit{with} replacement,
    \item Regression adjustment with replacement: $ \bar{f} + \frac1m \sum_{i \in S} (y_i - f(\mathbf{x}_i)$ where samples in $S$ are drawn \textit{with} replacement,
    \item Monte Carlo without replacement: $ \frac1m \sum_{i \in S} y_i$ where samples in $S$ are drawn \textit{without} replacement, and
    \item Monte Carlo without replacement: $ \bar{f} + \frac1m \sum_{i \in S} (y_i - f(\mathbf{x}_i)$ where samples in $S$ are drawn \textit{without} replacement.
\end{itemize}
For variance budgets $m \leq n$, plot the MSE between each estimate and the mean.
The horizontal access should be $m$, and the vertical access should be the MSE.

%\input{solutions/solution12_1}

\newpage
\section*{Problem 2: SHAP}

\subsection*{Part A: Data, Training, SHAP}

Using \texttt{shap}, load (a subset of) the California dataset.
Using \texttt{sklearn}, train a linear regression and neural network model on the data.
Using \texttt{shap}, apply an explainer of your choice to each model and the dataset.

\subsection*{Part B: Waterfall Plot}

For the same observation in the dataset, make a waterfall plot with the Shapley values for both models. What do you notice?

\subsection*{Part C: Beeswarm Plot}
Make a beeswarm plot with the Shapley values for both models. What do you notice?

%\input{solutions/solution12_2}

\end{document}