\documentclass{article}
\usepackage[utf8]{inputenc}
\usepackage[a4paper, total={6in, 8in}]{geometry}
\usepackage{amsmath, amsfonts}
\usepackage{hyperref, graphicx}
\usepackage{tikz, bbm, mathtools}

\DeclareMathOperator*{\argmin}{arg\,min}
\DeclareMathOperator*{\argmax}{arg\,max}

\title{CSCI 145 Problem Set 11}
\author{} % TODO: Put your name here
\date{\today}

\begin{document}

\maketitle

\subsection*{Submission Instructions}

Please upload \textit{your} work by
\textbf{11:59pm Monday November 17, 2025.}
\begin{itemize}
\item You are encouraged to discuss ideas
and work with your classmates. However, you
\textbf{must acknowledge} your collaborators
at the top of each solution on which
you collaborated with others 
and you \textbf{must write} your solutions
independently.
\item Your solutions to theory questions must
be written legibly, or typeset in LaTeX or markdown.
If you would like to use LaTeX, you can import the source of this document (available from the course webpage) to Overleaf.
\item I recommend that you write your solutions to coding questions in a Jupyter notebook using Google Colab.
\item You should submit your solutions as a \textbf{single PDF} via the assignment on Gradescope.
\end{itemize}

\noindent
\textbf{Grading:} The point of the problem set is for \textit{you} to learn. To this end, I hope to disincentivize the use of LLMs by \textbf{not} grading your work for correctness. Instead, you will grade your own work by comparing it to my solutions. This self-grade is due the Friday \textit{after} the problem set is due, also on Gradescope.

\newpage
\section*{Problem 1: Concentration}

Consider $n$ \textit{independent} random variables $X_1, \ldots, X_n$.
Each random variable is bounded between 0 and 1, i.e., $X_i \in [0,1]$
Define the empirical mean
\begin{align}
    \bar{X} = \frac1n \sum_{i=1}^n X_i.
\end{align}

In this problem, we will analyze $\bar{X}$.

\subsection*{Part A: Variance}

Prove that $\text{Var}(\bar{X}) \leq \frac1n$.

\textbf{Hint:} Use the linearity of variance for independent random variables.

\subsection*{Part B: Concentration Inequalities}

Apply Markov's, Chebyshev's, and Hoeffding's inequality to bound how $\bar{X}$ deviates from its mean $\mathbb{E}[\bar{X}]$.
Rearrange the inequalities so that the probabilities are at most $\frac1{\epsilon}, \frac1{\epsilon^2}$, and $\frac1{e^\epsilon}$ respectively.

Note: Use part A when manipulating Chebyshev's inequality.

\subsection*{Part C: Union Bound}

In the prior part, we saw that---except for the probability bound---Chebyshev's is quite similar to Hoeffding's.
Consider $m$ different empirical means $\bar{X}^{(1)}, \ldots, \bar{X}^{(m)}$, each with their own $n$ independent random variables $X_1^{(j)}, \ldots, X_n^{(j)}$ for $j=1, \ldots, m.$

Use Hoeffding's and the Union Bound to upper bound the probability that, for \textit{any} $j$, we have
\begin{align}
\bar{X}^{(j)} -\mathbb{E}[\bar{X}^{(j)}] > \frac{\epsilon}{\sqrt{2n}}.
\end{align}
How should we set $\epsilon$ if we want this probability to be $\frac1{100}$?


%\input{solutions/solution11_1}

\newpage
\section*{Multi-armed Bandits}

In this problem, we will explore some of the ideas from lecture in more detail.
Before working on these problems, familiarize yourself with the proof of the regret theorem.

\subsection*{Part A: Integral Upper Bound}
In the proof of the regret theorem, we used several elegant mathematical bounds.
Prove that
\begin{align}
    \sum_{i=1}^n \frac1{\sqrt{i}}
    \leq 2\sqrt{n}.
\end{align}
Hint: Upper bound the summation with an integral $\int f(x) dx$ for an appropriately chosen function $f$.

\subsection*{Part B: Cauchy-Schwarz Inequality}
The Cauchy-Schwarz inequality is an incredibly useful tool.
For vectors $\mathbf{a}, \mathbf{b} \in \mathbb{R}^k$, Cauchy-Schwarz tells us that
\begin{align}
    \langle \mathbf{a}, \mathbf{b} \rangle^2 \leq \| \mathbf{a} \|_2^2 \| \mathbf{b} \|_2^2.
\end{align}
Recall that $\sum_{a=1}^k n_{a}^{(T)}=T $.
Use Cauchy-Schwarz to show that
\begin{align}
    \sum_{a=1}^k \sqrt{n_{a}^{(T)}} \leq \sqrt{k T}.
\end{align}

\subsection*{Part C: Another Agent}

Develop another agent for solving the multi-armed bandit problem.
Using the code from class, implement your agent and compare its performance to the greedy, random, and UCB agents we tried.

%\input{solutions/solution11_2}

\end{document}